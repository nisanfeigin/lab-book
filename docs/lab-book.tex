\documentclass[]{book}
\usepackage{lmodern}
\usepackage{amssymb,amsmath}
\usepackage{ifxetex,ifluatex}
\usepackage{fixltx2e} % provides \textsubscript
\ifnum 0\ifxetex 1\fi\ifluatex 1\fi=0 % if pdftex
  \usepackage[T1]{fontenc}
  \usepackage[utf8]{inputenc}
\else % if luatex or xelatex
  \ifxetex
    \usepackage{mathspec}
  \else
    \usepackage{fontspec}
  \fi
  \defaultfontfeatures{Ligatures=TeX,Scale=MatchLowercase}
\fi
% use upquote if available, for straight quotes in verbatim environments
\IfFileExists{upquote.sty}{\usepackage{upquote}}{}
% use microtype if available
\IfFileExists{microtype.sty}{%
\usepackage{microtype}
\UseMicrotypeSet[protrusion]{basicmath} % disable protrusion for tt fonts
}{}
\usepackage{hyperref}
\hypersetup{unicode=true,
            pdftitle={Tirosh Lab Book},
            pdfborder={0 0 0},
            breaklinks=true}
\urlstyle{same}  % don't use monospace font for urls
\usepackage{natbib}
\bibliographystyle{apalike}
\usepackage{color}
\usepackage{fancyvrb}
\newcommand{\VerbBar}{|}
\newcommand{\VERB}{\Verb[commandchars=\\\{\}]}
\DefineVerbatimEnvironment{Highlighting}{Verbatim}{commandchars=\\\{\}}
% Add ',fontsize=\small' for more characters per line
\usepackage{framed}
\definecolor{shadecolor}{RGB}{248,248,248}
\newenvironment{Shaded}{\begin{snugshade}}{\end{snugshade}}
\newcommand{\KeywordTok}[1]{\textcolor[rgb]{0.13,0.29,0.53}{\textbf{#1}}}
\newcommand{\DataTypeTok}[1]{\textcolor[rgb]{0.13,0.29,0.53}{#1}}
\newcommand{\DecValTok}[1]{\textcolor[rgb]{0.00,0.00,0.81}{#1}}
\newcommand{\BaseNTok}[1]{\textcolor[rgb]{0.00,0.00,0.81}{#1}}
\newcommand{\FloatTok}[1]{\textcolor[rgb]{0.00,0.00,0.81}{#1}}
\newcommand{\ConstantTok}[1]{\textcolor[rgb]{0.00,0.00,0.00}{#1}}
\newcommand{\CharTok}[1]{\textcolor[rgb]{0.31,0.60,0.02}{#1}}
\newcommand{\SpecialCharTok}[1]{\textcolor[rgb]{0.00,0.00,0.00}{#1}}
\newcommand{\StringTok}[1]{\textcolor[rgb]{0.31,0.60,0.02}{#1}}
\newcommand{\VerbatimStringTok}[1]{\textcolor[rgb]{0.31,0.60,0.02}{#1}}
\newcommand{\SpecialStringTok}[1]{\textcolor[rgb]{0.31,0.60,0.02}{#1}}
\newcommand{\ImportTok}[1]{#1}
\newcommand{\CommentTok}[1]{\textcolor[rgb]{0.56,0.35,0.01}{\textit{#1}}}
\newcommand{\DocumentationTok}[1]{\textcolor[rgb]{0.56,0.35,0.01}{\textbf{\textit{#1}}}}
\newcommand{\AnnotationTok}[1]{\textcolor[rgb]{0.56,0.35,0.01}{\textbf{\textit{#1}}}}
\newcommand{\CommentVarTok}[1]{\textcolor[rgb]{0.56,0.35,0.01}{\textbf{\textit{#1}}}}
\newcommand{\OtherTok}[1]{\textcolor[rgb]{0.56,0.35,0.01}{#1}}
\newcommand{\FunctionTok}[1]{\textcolor[rgb]{0.00,0.00,0.00}{#1}}
\newcommand{\VariableTok}[1]{\textcolor[rgb]{0.00,0.00,0.00}{#1}}
\newcommand{\ControlFlowTok}[1]{\textcolor[rgb]{0.13,0.29,0.53}{\textbf{#1}}}
\newcommand{\OperatorTok}[1]{\textcolor[rgb]{0.81,0.36,0.00}{\textbf{#1}}}
\newcommand{\BuiltInTok}[1]{#1}
\newcommand{\ExtensionTok}[1]{#1}
\newcommand{\PreprocessorTok}[1]{\textcolor[rgb]{0.56,0.35,0.01}{\textit{#1}}}
\newcommand{\AttributeTok}[1]{\textcolor[rgb]{0.77,0.63,0.00}{#1}}
\newcommand{\RegionMarkerTok}[1]{#1}
\newcommand{\InformationTok}[1]{\textcolor[rgb]{0.56,0.35,0.01}{\textbf{\textit{#1}}}}
\newcommand{\WarningTok}[1]{\textcolor[rgb]{0.56,0.35,0.01}{\textbf{\textit{#1}}}}
\newcommand{\AlertTok}[1]{\textcolor[rgb]{0.94,0.16,0.16}{#1}}
\newcommand{\ErrorTok}[1]{\textcolor[rgb]{0.64,0.00,0.00}{\textbf{#1}}}
\newcommand{\NormalTok}[1]{#1}
\usepackage{longtable,booktabs}
\usepackage{graphicx,grffile}
\makeatletter
\def\maxwidth{\ifdim\Gin@nat@width>\linewidth\linewidth\else\Gin@nat@width\fi}
\def\maxheight{\ifdim\Gin@nat@height>\textheight\textheight\else\Gin@nat@height\fi}
\makeatother
% Scale images if necessary, so that they will not overflow the page
% margins by default, and it is still possible to overwrite the defaults
% using explicit options in \includegraphics[width, height, ...]{}
\setkeys{Gin}{width=\maxwidth,height=\maxheight,keepaspectratio}
\IfFileExists{parskip.sty}{%
\usepackage{parskip}
}{% else
\setlength{\parindent}{0pt}
\setlength{\parskip}{6pt plus 2pt minus 1pt}
}
\setlength{\emergencystretch}{3em}  % prevent overfull lines
\providecommand{\tightlist}{%
  \setlength{\itemsep}{0pt}\setlength{\parskip}{0pt}}
\setcounter{secnumdepth}{5}
% Redefines (sub)paragraphs to behave more like sections
\ifx\paragraph\undefined\else
\let\oldparagraph\paragraph
\renewcommand{\paragraph}[1]{\oldparagraph{#1}\mbox{}}
\fi
\ifx\subparagraph\undefined\else
\let\oldsubparagraph\subparagraph
\renewcommand{\subparagraph}[1]{\oldsubparagraph{#1}\mbox{}}
\fi

%%% Use protect on footnotes to avoid problems with footnotes in titles
\let\rmarkdownfootnote\footnote%
\def\footnote{\protect\rmarkdownfootnote}

%%% Change title format to be more compact
\usepackage{titling}

% Create subtitle command for use in maketitle
\providecommand{\subtitle}[1]{
  \posttitle{
    \begin{center}\large#1\end{center}
    }
}

\setlength{\droptitle}{-2em}

  \title{Tirosh Lab Book}
    \pretitle{\vspace{\droptitle}\centering\huge}
  \posttitle{\par}
    \author{}
    \preauthor{}\postauthor{}
      \predate{\centering\large\emph}
  \postdate{\par}
    \date{Last updated on 2019-11-06}

\usepackage{booktabs}
\usepackage{amsthm}
\makeatletter
\def\thm@space@setup{%
  \thm@preskip=8pt plus 2pt minus 4pt
  \thm@postskip=\thm@preskip
}
\makeatother

\begin{document}
\maketitle

{
\setcounter{tocdepth}{1}
\tableofcontents
}
\chapter{Homepage}\label{homepage}

\section{Links}\label{links}

\subsection{Useful}\label{useful}

\begin{itemize}
\tightlist
\item
  \href{https://www.dropbox.com/home/Tirosh\%20Lab}{Dropbox home}\\
\item
  \href{http://www.weizmann.ac.il/mcb/tirosh/publications-0}{Lab
  publications}\\
\item
  \href{http://www.weizmann.ac.il/mcb/tirosh/}{Lab website}\\
\item
  \href{https://www.dropbox.com/sh/javjoi26i5k5f6t/AADAX4RgSZUFuBwtS6TRE3wua?dl=0}{Experimental
  protocols}
\end{itemize}

\subsection{Fun}\label{fun}

\begin{itemize}
\tightlist
\item
  \href{https://twitter.com/tiroshlab?lang=bg}{Lab Twitter}\\
\item
  \href{https://www.dropbox.com/s/zwss5fg9rkoz7pp/lab\%20bday\%20list.xlsx?dl=0}{Birthdays}\\
\item
  \href{http://www.weizmann.ac.il/mcb/tirosh/group-members}{Lab
  members}\\
\item
  \href{http://www.weizmann.ac.il/mcb/tirosh/pictures}{Lab trip photos}
\end{itemize}

\section{Adding to the book}\label{adding-to-the-book}

Adding to the lab book is easy. The only requirement is that you
\href{https://github.com/join?source=header-home}{make a Github
account}, which shouldn't take more than a couple of minutes to do.

Note that since the book is re-built with each edit made, \textbf{you
may have to wait a few minutes before your changes become visible}.

\subsection{Option 1}\label{option-1}

\begin{enumerate}
\def\labelenumi{\arabic{enumi}.}
\tightlist
\item
  Go to the website chapter that you would like to edit.

  \begin{itemize}
  \tightlist
  \item
    E.g.
    \href{https://tiroshlab.github.io/lab-book/contact.html}{Chapter 3
    Lab members and contact details}
  \end{itemize}
\item
  Click on the edit icon at the top of the website page.
\item
  Make your edits and submit them (green button at the bottom of the
  page).
\end{enumerate}

\subsection{Option 2}\label{option-2}

\textbf{Note}: this will only work for those whose Github users have
been added to the book repository.

\begin{enumerate}
\def\labelenumi{\arabic{enumi}.}
\tightlist
\item
  Go to your local clone of the repository.

  \begin{itemize}
  \tightlist
  \item
    to clone the repository:
    \texttt{git\ clone\ https://github.com/tiroshlab/lab-book.git}\\
  \end{itemize}
\item
  Make and save the relevant changes to the .Rmd file in question.

  \begin{itemize}
  \tightlist
  \item
    There is one .Rmd file per chapter.
  \end{itemize}
\item
  Submit the edit. In the terminal type:

  \begin{itemize}
  \tightlist
  \item
    \texttt{git\ commit\ -am\ \textless{}myShortMessage\textgreater{}}
  \item
    \texttt{git\ pull\ origin\ master}
  \item
    \texttt{git\ push\ origin\ master}
  \end{itemize}
\end{enumerate}

\section{Book formats}\label{book-formats}

The default book is a
\href{https://tiroshlab.github.io/lab-book}{website}. You can also
choose to download the book as PDF or ebook. To do this, click the
download icon at the top of the website page.

\chapter{General research directions of the lab}\label{general}

We are a cancer systems biology lab, focused on intra-tumor
heterogeneity (ITH). Our main goal is to understand ITH, the mechanisms
that generate and maintain it, and the functional and clinical
implications.

\section{ITH mechanisms}\label{ith-mechanisms}

We consider three main types of mechanisms: 1. Genetic heterogeneity
(genetic subclones). 2. Microenvironmental influences
(e.g.~oxygen/nutrient availability and cell-cell interactions) 3.
Intrinsic cellular plasticity (cells dynamically transitioning between
states).

\section{ITH functional/clinical
implications}\label{ith-functionalclinical-implications}

We are interested in all implications of ITH but focus primarily on the
questions of drug resistance, cancer stem cells and metastasis/invasion,
all of which have highly significant clinical implications.

\section{Approach}\label{approach}

We use a systems biology approach, which means that we combine
experimental and computational approaches to address these questions
from a global perspective, typically through single cell RNA-seq or
other high-throughput profiling approaches, followed by extensive
computational analyses, and experimental validations.

\section{Experimentally}\label{experimentally}

We emphasize the ability to work directly with patient samples and to
that end we work closely with clinicians. In addition, we work with
``simple'' model systems (cancer cell lines), but only after we verify
that they recapitulate certain aspects of heterogeneity as seen in
tumors. We tend to avoid working with complex models and with
experimental systems that are difficult and slow to work with, such as
mice models, although this is possible through collaborations with other
labs. Our main experimental method is scRNA-seq, but we try to combine
it and follow up with complementary approaches (FACS analysis, spatial
analyses, time-course experiments, ATAC-seq) which will expand and
evolve with time.

\section{Computationally}\label{computationally}

We emphasize a biology-centered and data-driven approach, and hence
perform hands-on extensive, iterative, and integrative data analyses
while avoiding the ``blind'' use of computational tool. We do not focus
on any particular computational methodology but attempt to use the
appropriate method for each analysis, while keeping the analysis
intuitive and simple.

\section{Biological contexts}\label{biological-contexts}

We study primarily glioma and head and neck cancer. These are long-term
directions in which we collaborate closely with experimental/clinical
groups (Mario Suva from Boston and Sid Puram from St.~Louis), striving
to deeply understand the underlying biology and ultimately make a
clinical impact. We are currently starting also a third focus on
neuroendocrine tumors (NETs), in collaboration with Amit Tirosh from
Sheba Medical Center (no family relations). In addition, specific
projects involve other contexts due to either exceptional diversity
(e.g.~Carcinosarcomas: tumors that contain both carcinoma and sarcoma
cells), or collaborative efforts (e.g.~in ovarian and pancreatic
cancer).

\chapter{Lab members and contact details}\label{contact}

\begin{enumerate}
\def\labelenumi{\arabic{enumi}.}
\tightlist
\item
  Itay Tirosh,
  \href{mailto:tirosh.itay@gmail}{\nolinkurl{tirosh.itay@gmail}},
  054-9453547, Hanasi Harishon 54 apt. 20, Rehovot
\item
  Alissa Greenwald,
  \href{mailto:aliscohen@gmail.com}{\nolinkurl{aliscohen@gmail.com}},
  054-5761984, Uri Zvi Grinberg 1A, Rehovot
\item
  Rony Chanoch,
  \href{mailto:ronychanoch@gmail.com}{\nolinkurl{ronychanoch@gmail.com}},
  050-5982514, Moshe Sharet 3 apt 11, Tel Aviv
\item
  Mike Tyler,
  \href{mailto:m20tyler@gmail.com}{\nolinkurl{m20tyler@gmail.com}},
  058-6594177, Derekh Yavne 81 apt. 25, Rehovot
\item
  Avishay Spitzer,
  \href{mailto:dr.avishayspitzer@gmail.com}{\nolinkurl{dr.avishayspitzer@gmail.com}},
  052-8365141, Hameri 18, apt. 3, Givatayim
\item
  Michael Mints,
  \href{mailto:michael.mints@umu.se}{\nolinkurl{michael.mints@umu.se}},
  058-5734784, Sharabi 2, Tel Aviv
\item
  Julie Laffy,
  \href{mailto:jlaffy7@gmail.com}{\nolinkurl{jlaffy7@gmail.com}},
  058-7075989, Dizengoff 149 apt. 3, Tel Aviv
\item
  Noam Hadary,
  \href{mailto:noam.hadary@weizmann.ac.il}{\nolinkurl{noam.hadary@weizmann.ac.il}},
  052-4323212, Derech Yavne 63, apt. 14, Rehovot
\end{enumerate}

\chapter{Specific Projects}\label{projects}

\textbf{Alissa}\\
My two main projects are 1) Modeling ITH in cancer cell lines with a
focus on dynamics, differential drug sensitivity, and cancer cell-cancer
cell interactions in HNSCC (with Rotem). 2) Spatial transcriptomics of
GBM (in collaboration with the Suva Lab).

\textbf{Rony}\\
My thesis focuses on GBM intra-tumor heterogeneity, for which I have 3
projects in collaboration with Mario Suva's lab at MGH: 1) modeling the
GBM states in cancer cell lines and a mouse model of GBM, 2) exploring
the GBM microenvironment cell-cell interactions in regards to the
cellular states, and 3) targeting the GBM states as a therapeutic
approach for GBM.

\textbf{Mike}\\
My work concerns deconvolution of bulk RNA-seq data to identify
signatures of tumour subpopulations or cell states. My current project
focuses on distinguishing EMT (or partial EMT) from stromal cells in
bulk transcriptomes.

\textbf{Michael}\\
My work broadly focuses on ITH in head and neck cancer. In collaboration
with Sid Puram's lab I analyse scRNASeq data from patients with
oropharyngeal, laryngeal and oral cavity cancer in order to find tumour
genetic subclones and functional metaprograms in the cancer cells and
microenvironment.

\textbf{Noam H}\\
My work is to do a regulatory follow up to lab discoveries on chromatin
organization level using single cell ATAC-seq on two main ITH projects:
1) head and neck cancer 2) GBM.

\textbf{Julie}\\
My work focuses on understanding expression heterogeneity in gliomas. I
analyse primarily scRNA-seq data from the Suva lab to 1) define the
subpopulations of cancer cells that exist in different glioma types, 2)
understand the emergence of these subpopulations as a function of cell
lineage, genetics and the TME and 3) revisit the distinctions between
glioma types.

\chapter{Joining and Leaving protocol}\label{joiningLeaving}

\section{Joining}\label{joining}

When joining the group:

\begin{enumerate}
\def\labelenumi{\arabic{enumi}.}
\tightlist
\item
  Ask Michal Ovadia to pool your userID to this department and connect
  you to any services you may need to use (at least WEXAC and email will
  be necessary, and RSA very useful), note, if you have userID in other
  department, pooling you will erase your old server folder.
\item
  Ask Michal to add you to the lab dropbox.
\item
  Order a computer -- available options are found at the
  \href{http://www.weizmann.ac.il/DIS/computing-hardware/pc/windows-laptops}{DIS
  site} Check with Itay before ordering.
\item
  Natali Yikne is the one that supplies network cable, opens the network
  connection point near to your desk and makes your registration to the
  internet - for that, you will have to supply your ETHERNET code (found
  on the Ethernet outlet) and the computer's MAC address.
\item
  Ask Itay to get the lab google calendar so you can update and know
  future plans.
\item
  If you're a foreigner, go to the FGS for help to set up banking and
  insurance.
\item
  Also as a foreigner you can
  \href{http://www.weizmann.ac.il/vs/}{arrange housing through the
  visiting scientist centre}.
\item
  Provide your bank account to Michal so you can get paid.
\item
  After registering you will get emails about doing a medical checkup,
  lab safety introduction and having your photo taken so you can get a
  Weizmann badge.
\end{enumerate}

\section{Leaving}\label{leaving}

When leaving the group:

\begin{enumerate}
\def\labelenumi{\arabic{enumi}.}
\tightlist
\item
  Make sure to return all your computer accessories (docking station,
  charger, mouse, keyboard etc)
\item
  Make sure to back up your code, results and data, as your folder will
  be erased.
\item
  Talk to Michal to remove you from services.
\item
  Return your badge to Michal when you leave.
\item
  Notify Natali whether you possess Weizmann laptop and accessories and
  who will get that after you.
\end{enumerate}

\chapter{Lab Responsibilities}\label{labResponsibilities}

\begin{enumerate}
\def\labelenumi{\arabic{enumi}.}
\item
  Ordering (scientific, food)
\item
  Group meetings and journal club organization.

  \begin{itemize}
  \tightlist
  \item
    Julie
  \item
    Mike
  \end{itemize}
\item
  Lab trip and lab week organization

  \begin{itemize}
  \tightlist
  \item
    Rony
  \end{itemize}
\item
  Equipment/reagents maintenance
\item
  Lab code

  \begin{itemize}
  \tightlist
  \item
    Julie
  \item
    Avishay
  \end{itemize}
\item
  Lab organization/setup (esp. including when moving to the new
  building)
\item
  Anything else I didn't consider
\end{enumerate}

\chapter{Shared Data}\label{sharedData}

\section{Datasets}\label{datasets}

Single-cell RNA-seq datasets that are commonly used by the lab are on
the server home under \texttt{shared/datasets}. Each dataset is a
separate .txt file and the files are categorised into subfolders by
cancer type (or ``Normal'').

For more information on the datasets found here, go to
\texttt{shared/datasets/INFO}.If you would like to add a dataset, please
also add the relevant metadata to the excel spreadsheet in the
\texttt{shared/datasets/INFO} directory.

\section{Gene signatures}\label{gene-signatures}

Gene signatures derived from scRNA-seq datasets are also on the server
home. You can find these under \texttt{shared/datasets/SIGNATURES}. The
signatures are collapsed into a single .txt file in two different
formats, ``long'' and ``wide''. The former includes metadata, while the
latter contains the minimal amount of information - one signature per
column of the matrix.

For more information on the gene signatures as well as how to add
signatures, please see \texttt{shared/datasets/SIGNATURES/README}.

\chapter{Shared Code}\label{sharedCode}

The aim of the shared code is to implement the lab's core ideas on
analysis of scRNA-seq data. The code is written in R and is publicly
available via GitHub (see below). It was designed with a modular
approach and hence is separated into several well-defined packages. The
packages and functions are well documented and examples are provided
within each package.

\section{About the R packages}\label{about-the-r-packages}

\subsection{\texorpdfstring{\texttt{scandal}}{scandal}}\label{scandal}

\href{https://github.com/dravishays/scandal}{Source code} \textbar{}
\href{https://github.com/dravishays/scandal/issues}{Report bugs}

\emph{\textbf{A framework that enables defining a single-cell
experiment.}}

The package provides methods for loading the data, preprocessing and
quality control, maintaining the data with a low memory footprint (using
sparse matrices), various plotting methods, linking meta-data with
expression data and more.

The package extends the \texttt{SingleCellExperiment} class, adapting it
for use in our lab. See
\href{https://www.bioconductor.org/packages/release/bioc/vignettes/SingleCellExperiment/inst/doc/intro.html}{this
tutorial} for an introduction to the \texttt{SingleCellExperiment}
class.

To install in R:

\begin{Shaded}
\begin{Highlighting}[]
\NormalTok{devtools}\OperatorTok{::}\KeywordTok{install_github}\NormalTok{(}\StringTok{"dravishays/scandal"}\NormalTok{)}
\end{Highlighting}
\end{Shaded}

\subsection{\texorpdfstring{\texttt{infercna}}{infercna}}\label{infercna}

\href{https://jlaffy.github.io/infercna}{Website \& Tutorials}
\textbar{}
\href{https://jlaffy.github.io/infercna/reference/index.html}{Functions
index} \textbar{} \href{https://github.com/jlaffy/infercna}{Source code}
\textbar{} \href{https://github.com/jlaffy/infercna/issues}{Report bugs}

\emph{\textbf{Infer copy-number alterations from (single-cell)
RNA-sequencing data.}}

The methodology implemented here was first formulated by Itay and
colleagues during his postdoc
\href{https://science.sciencemag.org/content/344/6190/1396.long}{Tirosh
et al., 2014} and has been tried and tested in several publications
since
(\href{https://science.sciencemag.org/content/360/6386/331.long}{Filbin
et al., 2018};
\href{https://www.ncbi.nlm.nih.gov/pubmed/31327527}{Neftel et al.,
2019}; \href{https://www.ncbi.nlm.nih.gov/pubmed/29198524}{Puram et al.,
2017}; Tirosh et al.,
\href{https://science.sciencemag.org/content/352/6282/189.long}{2016a},
\href{https://www.nature.com/articles/nature20123}{2016b};
\href{https://science.sciencemag.org/content/355/6332/eaai8478.long}{Venteicher
et al., 2017}).

To install in R:

\begin{Shaded}
\begin{Highlighting}[]
\NormalTok{devtools}\OperatorTok{::}\KeywordTok{install_github}\NormalTok{(}\StringTok{"jlaffy/infercna"}\NormalTok{)}
\end{Highlighting}
\end{Shaded}

\href{https://jlaffy.github.io/infercna/articles/useGenome.html}{Tutorial
1: Set your genome}\\
\href{https://jlaffy.github.io/infercna/articles/infercna_tutorial.html}{Tutorial
2: Example with a scRNA-seq dataset}

\subsection{\texorpdfstring{\texttt{scrabble}}{scrabble}}\label{scrabble}

\href{https://jlaffy.github.io/scrabble}{Website \& Tutorials}
\textbar{}
\href{https://jlaffy.github.io/scrabble/reference/index.html}{Functions
index} \textbar{} \href{https://github.com/jlaffy/scrabble}{Source code}
\textbar{} \href{https://github.com/jlaffy/scrabble/issues}{Report bugs}

\emph{\textbf{Perform exploratory computational analyses on processed
scRNAseq gene expression data.}}

The package focuses on unbiased methods in unsupervised clustering and
dimensionality reduction to identify and characterize the
transcriptionally-distinct subpopulations of cancer cells residing
within tumours.

In its current implementation scrabble most closely reflects the methods
implemented in
\href{https://www.ncbi.nlm.nih.gov/pubmed/31327527}{Neftel et al., 2019}
though any of the lab's papers should be useful as reference.

To install in R:

\begin{Shaded}
\begin{Highlighting}[]
\NormalTok{devtools}\OperatorTok{::}\KeywordTok{install_github}\NormalTok{(}\StringTok{"jlaffy/scrabble"}\NormalTok{)}
\end{Highlighting}
\end{Shaded}

\section{Contributing}\label{contributing}

Members of the lab to whom the shared code is relevant and of use are
encouraged to contribute to it. This will both help the code to grow and
develop but also allow you to fine-tune it for your own analyses. To
make individual contributions, you can
\href{https://help.github.com/en/github/getting-started-with-github/fork-a-repo}{fork
the project from GitHub}. To contribute in a more long-term way, ask
Julie or Avishay to add you as an official contributor to the GitHub
package(s) in question.

\chapter{Experimental Protocols}\label{protocols}

A set of experimental protocols can be found in the Tirosh Lab dropbox
organized by subcategory
(\href{https://www.dropbox.com/sh/javjoi26i5k5f6t/AADAX4RgSZUFuBwtS6TRE3wua?dl=0}{link
to folder}). Please add your own protocols so this can become an
improved and evolving resource for all experimentalists.

\chapter{Ordering}\label{ordering}

\section{Weizmann warehouse orders}\label{weizmann-warehouse-orders}

An ongoing list for the ordering of lab supplies from the Weizmann
warehouse (machsan) is kept on the whiteboard outside Itay's office.
Items ordered from the machsan include all disposables (i.e.~tubes,
gloves, pipettes), common/basic equipment (i.e timers, pipettors, lab
coats), most tissue culture reagents (i.e.~media, PBS), basic chemicals
(i.e.~ethanol, DMSO), and commonly used reagents (i.e.~kits for qPCR,
cDNA synthesis, RNA isolation). Warehouse orders are placed through the
Tafnit system. Alissa regularly places orders but Rotem and Rony also
have access to Tafnit (if anyone else would like access, please ask
Michal Ovadia; note that the system is only available in Hebrew). When
an item on the list is ordered, it should be marked with a checkmark.
When the item arrives, it should be erased from the list. Orders from
the machsan usually arrive within 3-4 days of being placed. If it is
urgent, it is possible to call the warehouse and arrange to pick up an
item in person from Piccioto after the order is received.

\section{General ordering}\label{general-ordering}

All items that aren't ordered from the warehouse can be ordered with a
price quote. In principle, some items from Sigma and Grafiti can also be
ordered via Tafnit, though it is usually easier to order via price
quote. Receiving a price quote requires knowing the local distributor of
the product - many company websites will have a list of international
distributors. Below, you will also find contact info for many of the
most relevant distributors.

To receive a price quote, email the relevant distributor with the
catalog number and manufacturer. You can also ask how long it will take
for the product to arrive if it's time sensitive (many products are
ordered from abroad). Include in the quote request or in your email
signature that you are from the Tirosh lab and lab address/departmental
affiliation. You can forward price quotes to Natali
(\href{mailto:natalie.yikne@weizmann.ac.il}{\nolinkurl{natalie.yikne@weizmann.ac.il}})
and she will place the order.

\section{Office supplies}\label{office-supplies}

Graffiti is the office supply distributor for the Weizmann Institute.
Their inventory/catalog can be found on
\href{https://www.graffiti-online.co.il/}{their website}. You can
receive a price quote from
\href{mailto:liatji@graf.co.il}{\nolinkurl{liatji@graf.co.il}}.

\section{Antibodies}\label{antibodies}

In the Tirosh Lab dropbox, there is an
\href{https://www.dropbox.com/s/8fr2p795g5zgepz/antibodies_Tirosh.xlsx?dl=0}{excel
file} listing the details of the antibodies we have in the lab. Please
consult it before ordering a new antibody and please update it when you
receive a new antibody. Biocompare.com is a useful site for finding
antibodies. You can filter your search with reactive species, conjugated
fluorophores, and by application (IHC, flow, etc). There are many
vendors of varying reputation for antibodies --- please ask if you have
questions about quality or which vendor to choose. For selecting
fluorophores for FACS antibodies, the
\href{https://www.bdbiosciences.com/en-us/applications/research-applications/multicolor-flow-cytometry/product-selection-tools/spectrum-viewer}{BD
spectrum viewer} is a helpful tool for selecting panels of antibodies
without spectral overlap.

Below are some recommended antibody vendors and their local
distributors.

For FACS antibodies:

\begin{itemize}
\tightlist
\item
  eBioscience (Rhenium), Biolegend (Enco), Miltenyi (Almog)
\end{itemize}

For immunohistochemistry/immunocytochemistry antibodies:

\begin{itemize}
\tightlist
\item
  Abcam (Zotal), R\&D (Biotest)
\end{itemize}

For price quotes:

\begin{itemize}
\tightlist
\item
  Rhenium (eBioscience, Thermofisher) - Yaara Willensky
  \href{mailto:yaara@rhenium.co.il}{\nolinkurl{yaara@rhenium.co.il}}
\item
  Almog (Miltenyi) - Aviva Blechman Peretz
  \href{mailto:aviva@almog.co.il}{\nolinkurl{aviva@almog.co.il}}
\item
  Biotest (R\&D) - Laora
  \href{mailto:laora@biotest.co.il}{\nolinkurl{laora@biotest.co.il}}
\item
  Zotal (Abcam) - Anna Bernstein
  \href{mailto:anna@zotal.co.il}{\nolinkurl{anna@zotal.co.il}}
\item
  Enco (Biolegend) - Yifat Ovadia -
  \href{mailto:Yifat@enco.co.il}{\nolinkurl{Yifat@enco.co.il}}
\end{itemize}

\section{Molecular biology}\label{molecular-biology}

Many popular molecular biology kits/products can be found in the
warehouse (for qPCR we are typically using Applied Biosystems FAST SYBR
green, for cDNA synthesis we like the Applied Biosystems high capacity
cDNA kit, both available from the warehouse). For RNA isolation, Qiagen
kits are available from the warehouse, though our favorite is the Zymo
Quick RNA micro prep kit (Zotal is the distributor).

\section{Growth factors}\label{growth-factors}

Please check the
\href{https://www.dropbox.com/s/q0dn3a3th12y822/cytokines_GFs_Tirosh.xlsx?dl=0}{growth
factors/cytokine spreadsheet} in the Tirosh Lab dropbox to see inventory
and please update when you receive something new. We have good
experience ordering growth factors/cytokines from Peprotech located in
Park HaMada (email
\href{mailto:ori@peprotechasia.com}{\nolinkurl{ori@peprotechasia.com}}
for quotes) and from Sino Biological (email Yifat Ovadia -
\href{mailto:Yifat@enco.co.il}{\nolinkurl{Yifat@enco.co.il}} for a
quote).

\section{Miltenyi/MACS products}\label{miltenyimacs-products}

Almog is the supplier for Miltenyi products including
reagents/disposables for the OctoMACS tissue dissociator, the dead cell
removal kit, and MACS beads/reagants. Email
\href{mailto:aviva@almog.co.il}{\nolinkurl{aviva@almog.co.il}} or
\href{mailto:netta@almog.co.il}{\nolinkurl{netta@almog.co.il}} for
quotes. There is an option to either place an individual order or to
join the monthly departmental order which saves money but takes longer.
If your order is non-urgent, you can request the FOB price and ask
Natali to add it to the monthly departmental order when you send her the
quote.

\section{Drugs}\label{drugs}

Cayman Chemicals is a reasonably priced and reliable supplier for
drugs/compounds. Their distributor is Enco
(\href{mailto:Yifat@enco.co.il}{\nolinkurl{Yifat@enco.co.il}} for
quotes). Other vendors for drugs/chemical compounds include Sigma/Merck
(\href{mailto:shir.yohai@merckgroup.com}{\nolinkurl{shir.yohai@merckgroup.com}}),
Molport
(\href{mailto:jose.garcia-tenorio@molport.com}{\nolinkurl{jose.garcia-tenorio@molport.com}}),
and Selleck
(\href{mailto:manager@tivanbiotech.com}{\nolinkurl{manager@tivanbiotech.com}}).

\section{Chemicals}\label{chemicals}

Sigma/Merck is the largest distributor for chemicals. Many common
chemicals can be ordered via the warehouse; Sigma price quote requests
can be sent to
\href{mailto:shir.yohai@merckgroup.com}{\nolinkurl{shir.yohai@merckgroup.com}}.

\section{Primers}\label{primers}

For primer design, everyone has their favorite program, but NCBI Primer
BLAST is a good place to start. We order primers through IDT/Syntezza
(idtdna.com). You can create your own user account with IDT on their
site. Since primers are inexpensive and we tend to place small orders,
we keep a standing order with IDT. In the payment order field, you can
write `Tirosh lab standing order'. You can email Chany Frankel
(\href{mailto:chany@syntezza.com}{\nolinkurl{chany@syntezza.com}}) to
confirm that there is money remaining in the standing order or to
request to add money to the standing order via a price quote.

\section{Illumina kits}\label{illumina-kits}

Danyel Biotech is the distributor for Illumina sequencing kits (request
a quote from
\href{mailto:sigal@danyel.co.il}{\nolinkurl{sigal@danyel.co.il}}). The
current kit in use for the NextSeq 500/550 is the NextSeq 500 high
output kit (75 cycles) FC-4-4-2005.

\chapter{Weizmann Services}\label{wisServices}

\section{Internal Services}\label{internal-services}

You can login to internal services using your Weizmann username/password
through the Weizmann website on a computer connected to the Weizmann
network or remotely with a VPN connection. Internal services functions
as an ordering/payment system for most services within the Institute.

\section{Printing Services}\label{printing-services}

For printing services, including printing of posters, you can make the
order in Internal Services. A PDF of the poster (or Dropbox link to PDF)
can be sent to
\href{mailto:fiana.parente@weizmann.ac.il}{\nolinkurl{fiana.parente@weizmann.ac.il}}.
It is also recommended to email or call Fiana to confirm receipt of the
order. Poster templates can be found in the Tirosh lab dropbox folder.
For more info on printing services, see
\href{https://www.weizmann.ac.il/RSD/units/design-photography-printing-branch/overview}{here}.

\section{VPN connection}\label{vpn-connection}

A VPN connection allows you to access the Weizmann network and the WEXAC
server remotely. You can request a VPN connection through RSA soft token
from Michal Ovadia. She will make the order to IT and IT will send
further instructions.

\section{Server (WEXAC) access/space}\label{server-wexac-accessspace}

You can request WEXAC access through Michal Ovadia and she will submit
the request to IT.

\section{Cryostorage}\label{cryostorage}

\section{Sending and Receiving
Packages}\label{sending-and-receiving-packages}

Fedex shipments can be coordinated through Michal Ovadia. Our Fedex
representative is Re'em Shpitzer
(\href{mailto:Rshiptzer@fdx.co.il}{\nolinkurl{Rshiptzer@fdx.co.il}},
054-560-6980) and our Fedex billing \# is 579301163. Shipments of
biological materials to/from abroad should include a declaration of
essence and a customs declaration. If temperature sensitive, be generous
with the dry ice.

\section{FACS unit}\label{facs-unit}

You should visit the unit in person to register and coordinate training.
Also, a safety form (to be renewed yearly) needs to be filled out and
submitted for each user and can be accessed under `FACS' in Internal
Services. After-hours access through your I.D. card can also be arranged
with the unit. Once you are a registered user, you can reserve equipment
through Internal Services. Visit the
\href{https://www.weizmann.ac.il/LS_CoreFacilities/flow-cytometry/about}{FACS
unit site} for further info on the unit, including the instruments.

\section{Sandbox unit}\label{sandbox-unit}

The Sandbox Unit is part of the Life Science Core Facilities located in
the Levine Building. The unit offers researchers access and support for
NGS-related technologies including library generation for bulk RNA-Seq
and 10X 3' scRNA-Seq as well as other 10X products (i.e.~ATAC-Seq). A
3-day workshop on generation of bulk RNA-Seq libraries with the bulk
MARS-Seq protocol is offered regularly. Unlike traditional core units,
you bring your samples and do the work yourself but all the materials
and equipment are located there. Reservations for bench space and
payment for use of supplies/reagants is via Genomics in Internal
Services. Contact Merav Kedmi
(\href{mailto:merav.kedmi@weizmann.ac.il}{\nolinkurl{merav.kedmi@weizmann.ac.il}},
x9212) or Hadas Keren-Shaul
(\href{mailto:hadas.keren-shaul@weizmann.ac.il}{\nolinkurl{hadas.keren-shaul@weizmann.ac.il}})
to arrange training, workshop attendance, and permissions. Additionally,
there is a Sandbox Whatsapp group for coordinating shared sequencing
runs (Hadas is admin).

There are also two NextSeq 500/550 sequencers in the unit. For training
and permissions, please contact Muriel Chemla
(\href{mailto:muriel.chemla@weizmann.ac.il}{\nolinkurl{muriel.chemla@weizmann.ac.il}})
in LSCF. Independent users can reserve sequencers through the Sandbox
reservation system in Internal Services. Note that we are not charged
for the sequencing run, only for the Illumina kit ordered through
Danyel. Initiating a sequencing run also requires an account in susanc
(ordered through the Bioinformatics Unit of LCSF, contact Irit Orr
\href{mailto:irit.orr@weizmann.ac.il}{\nolinkurl{irit.orr@weizmann.ac.il}}
x2470). See
\href{https://susanc.weizmann.ac.il/static/doc//howto.html}{this link}
for SampleSheet templates/info required for initiating a sequencing run.

\chapter{Budgets}\label{budgets}

Budgets beginning with 71- correspond to external grants and budgets
beginning with 72- correspond to internal grants. These budgets can be
used for most scientific supplies and services. Abisch Frankel (budget
713381) is only for kits for scRNA-Seq and for sequencing. It is
preferable to use external budgets first. Beyond that, priority is based
on expiration date and money remaining (best to use budgets with closest
expiration date and less money first). Equipment budgets begin with 4-.
For questions concerning budgets, contact Elinor David
(\href{mailto:elinor.david@weizmann.ac.il}{\nolinkurl{elinor.david@weizmann.ac.il}},
x4062).

Additionally, PhD students and postdocs have travel allowances. PhD
students receive \$2000 for travel following submission of their
research proposal and \$2250 following submission of their interim
report. For postdocs, the annual allowance is \$2000 yearly and the
calendar year starts on 1 October.

\chapter{Equipment and inventory}\label{equipment}

An excel sheet of the equipment found in the lab is in the Tirosh Lab
dropbox folder. Lumitron is the distributor who sold us the centrifuges
and the PCR machine and they can be contacted for service
(\href{mailto:adi.s@lumitron.co.il}{\nolinkurl{adi.s@lumitron.co.il}},
073-2000777 for service requests). Getter is the supplier/service
provider for the tissue culture hoods, the incubators, and the
fridges/freezers (Yuval Borenstein,
\href{mailto:yuvalb@getter.co.il}{\nolinkurl{yuvalb@getter.co.il}},
03-5761520 for service requests). Rhenium is the supplier/service
provider for the microscope and Countess (Hadar Adler,
\href{mailto:Hadar@rhenium.co.il}{\nolinkurl{Hadar@rhenium.co.il}}) and
the FACS analyzer
(\href{mailto:tech.support@rhenium.co.il}{\nolinkurl{tech.support@rhenium.co.il}},
08-955-8855 for service requests). Hay Cohen (08-934-344, Wolfson 408)
is the department administrator who oversees many equipment issues and
contacting him is a good starting place for equipment concerns.

\section{Office Supplies}\label{office-supplies-1}

Office supplies (pens, binders, notebooks, etc) can be found on the
shelves and in the drawers located by the printer.

\section{Lab supplies}\label{lab-supplies}

Extra stocks of disposables (i.e.~tissue culture plates and flasks,
pipette tips, eppendorfs) can be found in the closet in the tissue
culture room and in the closets with our lab name in the hall. These are
good places to check first if you think we may be out of something.

\section{Cytoflex (FACS analyzer)}\label{cytoflex-facs-analyzer}

Guidance and training on the Cytoflex can also be coordinated with
Carmit Hillel-Karniel
(\href{mailto:carmit@rhenium.co.il}{\nolinkurl{carmit@rhenium.co.il}}).
The sheath fluid consists of Millipore-grade water and 0.02\% sodium
azide. It can be either prepared by us or ordered through Rhenium. A
deep clean with Contrad should be performed monthly. A cleansing agent
with detergent is used for the daily clean. The blue tank is for sheath
fluid and the yellow tank is for waste. We have 3 lasers (blue, red,
violet) and the detector configuration can be viewed on the desktop of
the Cytoflex computer. A
\href{https://www.dropbox.com/s/ckvy0vshi45e95r/LSR\%20II\%20Quick\%20Reference.docx?dl=0}{word
doc outlining Cytoflex daily maintenance and use} is in the Tirosh Lab
dropbox.

\chapter{Reading list}\label{readingList}

A folder for lab recommended papers and other useful resources
(statistics,R, etc) is found on the lab dropbox at
\href{https://www.dropbox.com/sh/52ybes4a40ypn0q/AAAL2FlVKqRJw-Z4ScuWTW7Ra?dl=0}{this
link}. The folder is organized by topics which reflect the main
directions of the lab. The papers included are a short list of ``core''
papers that each lab member should read. Each new member of the lab
should start by reading the relevant papers to them. Any lab member can
recommend a paper to be added to the reading list, and this list is
managed by the chosen lab member that is responsible. In addition, in
the ``R'' folder, there is an excel sheet with recommended online
resources.

\chapter{Lab safety and Waste disposal}\label{safety}

\section{General Safety}\label{general-safety}

tl;dr x2999 or 08-934-2999 for emergencies and to report accidents
(24/7)

All lab members are required to attend a safety training. The powers at
be will likely contact you with a date to attend but if you choose to be
proactive you can email
\href{mailto:safety.training@weizmann.ac.il}{\nolinkurl{safety.training@weizmann.ac.il}}
for registration. :) A first aid kit is located on the top shelf by the
coffee machine (red bag). An emergency shower and eyewash station as
well as absorbent material for pouring on spills can be found
immediately outside the door of the main lab. All accidents should be
reported to x2999. Our annual lab safety plan (ASP) can be found in the
Tirosh Lab dropbox.

See
\href{https://www.weizmann.ac.il/safety/chemical-safety/general-guidelines}{here}
for detailed safety information.

\section{Waste disposal}\label{waste-disposal}

Chemical and biological waste can be disposed of in the chemical and
biological waste closet located in the opposite corridor on the first
floor of Wolfson. Chemical waste containers are located in the cabinet
above the sink in the main lab. Stickers for proper labeling are located
in the waste closet. Bags for biological waste are in the bottom to the
right of the sink in the main lab and in the bottom drawer in the tissue
culture room. Tape the bag shut before placing it in the biological
waste bin of the waste closet. Vacuum traps should be emptied before the
fluid level passes the line marked on them. They can be emptied into the
sink but should be refilled with 10\% volume of bleach after emptying.

\section{Chemical hood}\label{chemical-hood}

When working with hazardous chemicals, including Trizol and PFA, use the
hood located next door to the main lab in the common equipment room.
Please remember to clean up after yourself.

\section{Safety equipment}\label{safety-equipment}

Experimentalists should wear a lab coat, closed-toe shoes, and gloves
when conducting experiments. A face mask and insulated gloves for liquid
nitrogen are located in the main lab. Take special care when handling
human samples and avoid use of sharps when handling these samples if
possible. HBV vaccination is required for experimentalists handling
human samples.

\chapter{Lab Maintenance}\label{labMaintenance}

Ongoing chores and responsibilities to be divided among experimentalists
include:

\begin{enumerate}
\def\labelenumi{\arabic{enumi}.}
\tightlist
\item
  autoclave - preparation of materials (eppendorfs, Pasteur pipettes,
  H20) and transporting to/from autoclave on 3rd floor
\item
  waste disposal - bringing chemical and biological waste to the waste
  disposal closet (and replacing with new bags/containers), emptying
  vacuum traps in tissue culture and main lab and refilling with bleach
\item
  tissue culture maintenance - replacing water pans in the incubators
  and treating with Aquaguard (biweekly to monthly), replacing water in
  the water bath and treating with Aquaguard (monthly), autoclaving
  incubator racks (bi-annually), disinfection in case of a
  contamination, keeping the tissue culture room stocked with supplies
\item
  aliquoting of common reagents - aliquoting FBS, pen-strep, glutamine,
  Aquaguard 1\&2 on a regular basis for the whole lab and keeping these
  reagents in stock.
\end{enumerate}

\chapter{WEXAC (Weizmann EXAscale Cluster)}\label{wexac}

For more advanced help and information, please see the
\href{https://www.dropbox.com/s/uuv6qrkrwlhytrm/wexac_introduction.pdf?dl=0}{WEXAC
slides} on the lab dropbox.

\section{Software}\label{software}

\texttt{module\ avail} lists all available software.\\
\texttt{module\ avail} lists software if it exists, or versions of it if
they exist.\\
\texttt{module\ load} loads software.\\
\texttt{module\ ls} lists all software that you have currently loaded.

If you find yourself always loading the same modules, you may want to
instead add the relevant module load command to your shell's
configuration file in your home directory, e.g.~to
\textasciitilde{}/.bashrc. You can create this file if it doesn't exist.

\section{Adding software}\label{adding-software}

Option 1: Email
\href{mailto:HPC@wexac.weizmann.ac.il}{\nolinkurl{HPC@wexac.weizmann.ac.il}}
with a request to add software.\\
Option 2: Compile software in your home directory. Put the executables
in a sub-directory in your home and make sure to add the directory to
the \texttt{\$PATH} variable. E.g.
\texttt{export\ PATH=\textasciitilde{}/bin:\$PATH}, assuming you chose
to place executables in \texttt{\textasciitilde{}/bin}. To permanently
update the \$PATH variable, add the export command to your
\texttt{\textasciitilde{}/.bashrc}.

\section{Database}\label{database}

Some data are used by several labs at Weizmann and as such WEXAC have
put a significant amount of data under \texttt{/share/db}. You can check
whether for example your desired genome sequence or index file already
exists here before downloading it yourself.

\section{Our server on WEXAC}\label{our-server-on-wexac}

Our server or `compute node' is called \texttt{cn077}. It has
\textasciitilde{}200GB RAM and 1 core (Nov `19). \texttt{cn077} can be
accessed directly or via one of four `access nodes' (\texttt{access1},
\texttt{access2}, \texttt{access3}, \texttt{access4}) whose job it is to
properly manage and distribute memory between lab members. For this
reason, \textbf{if you're working interactively on the server you should
primarily use one of the access nodes}. It shouldn't matter which of the
four you use.

\textbf{Access}\\
access node:
\texttt{ssh\ \textless{}username\textgreater{}@access4.wexac.weizmann.ac.il}\\
compute node:
\texttt{ssh\ \textless{}username\textgreater{}@cn077.wexac.weizmann.ac.il}

\section{Submitting jobs}\label{submitting-jobs}

If you are running something that requires substantial memory/time, you
should submit this as a job to one of the queues on WEXAC (or sit back
and wait to receive a call from HPC).

To name a few queues:\\
\texttt{tirosh}\\
\texttt{new-short}\\
\texttt{new-medium}\\
\texttt{new-long}\\
\texttt{new-all}

Which queue you choose can depend on a number of things: job duration \&
memory requirements, how busy a queue is and which compute node it is
associated with. For example, the \texttt{tirosh} queue runs jobs on
\texttt{cn077}. Since \texttt{cn077} is private, the queue is not busy
(relatively speaking) and your job may start running sooner. On the
other hand since \texttt{cn077} only has 1 core, a
computationally-intensive job may finish more slowly than on a queue
with multiple cores.

For more information on available queues:\\
\texttt{bqueues\ -a}

To submit a job:\\
\texttt{bsub\ -q\ \textless{}queue\textgreater{}\ \textless{}job\textgreater{}}

Jobs are by default allocated 1GB of RAM. If this is not enough:\\
\texttt{bsub\ -q\ \textless{}queue\textgreater{}\ -R\ "rusage{[}mem=XGB{]}"\ \textless{}job\textgreater{}},
or\\
\texttt{bsub\ -q\ \textless{}queue\textgreater{}\ -R\ "select{[}mem\textgreater{}XGB{]}"\ \textless{}job\textgreater{}}\\
where \texttt{X} is the (minimum) amount of memory you want to allocate,
up to 150GB.

\texttt{bjobs} lists currently running and pending jobs.\\
\texttt{bkill\ \textless{}job\ ID\textgreater{}} kills job. Job ID is
given in \texttt{bjobs} output.\\
\texttt{bpeek} to glance at job progress (\texttt{STDOUT},
\texttt{STDERR}). See also \texttt{-o} and \texttt{-e} flags in
\texttt{bsub}.

\bibliography{book.bib,packages.bib}


\end{document}
